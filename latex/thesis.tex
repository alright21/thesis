%% LyX 2.3.3 created this file.  For more info, see http://www.lyx.org/.
%% Do not edit unless you really know what you are doing.
\documentclass[language=finnish,version=final,mainfont=none,sharelatex=false,hidechapters=true,countbibpages=false,emptyfirstpages=true]{utuftthesis}
\setcounter{secnumdepth}{2}
\setcounter{tocdepth}{2}
\usepackage{float}

\makeatletter

%%%%%%%%%%%%%%%%%%%%%%%%%%%%%% LyX specific LaTeX commands.
\providecommand\textquotedblplain{%
  \bgroup\addfontfeatures{Mapping=}\char34\egroup}
%% Because html converters don't know tabularnewline
\providecommand{\tabularnewline}{\\}
\floatstyle{ruled}
\newfloat{algorithm}{tbp}{loa}
\providecommand{\algorithmname}{Algoritmi}
\floatname{algorithm}{\protect\algorithmname}

\@ifundefined{showcaptionsetup}{}{%
 \PassOptionsToPackage{caption=false}{subfig}}
\usepackage{subfig}
\makeatother

\renewcommand{\listingscaption}{Ohjelmalistaus}

\addbibresource{Bibliografia.bib}
\begin{document}
\begin{comment}
Document template suitable for use as a LaTeX master-file for master's
thesis in University of Turku Department of Future Technologies.\\
\\
Compatible with: ShareLaTeX, pdfLaTeX, LuaLaTeX, XeLaTeX, LyX, latexmk.
The document class is described in utuftthesis.cls.\\
\\
HOW TO USE? See https://gitlab.utu.fi/ttweb/thesis
\end{comment}


\pubyear{2019}

\pubmonth{6}

\publab{Labran nimi}

\publaben{Laboratory Name}

\pubtype{tkk}
\title{Name of Thesis}
\author{My Name}

\maketitle
% main language keywords
\keywords{tähän, lista, avainsanoista}

% Second set of keywords (english)
% (in case the document is written in non-english)
\keywordstwo{here, a, list, of, keywords}

\begin{abstract}
Tarkempia ohjeita tiivistelmäsivun laadintaan läytyy opiskelijan yleisoppaasta,
josta alla lyhyt katkelma.

Bibliografisten tietojen jälkeen kirjoitetaan varsinainen tiivistelmä.
Sen on oletettava, että lukijalla on yleiset tiedot aiheesta. Tiivistelmän
tulee olla ymmärrettävissä ilman tarvetta perehtyä koko tutkielmaan.
Se on kirjoitettava täydellisinä virkkeinä, väliotsakeluettelona.
On käytettävä vakiintuneita termejä. Viittauksia ja lainauksia tiivistelmään
ei saa sisällyttää, eikä myäskään tietoja tai väitteitä, jotka eivät
sisälly itse tutkimukseen. Tiivistelmän on oltava mahdollisimman ytimekäs
n. 120--250 sanan pituinen itsenäinen kokonaisuus, joka mahtuu ykkäsvälillä
kirjoitettuna vaivatta tiivistelmäsivulle. Tiivistelmässä tulisi ilmetä
mm.  tutkielman aihe tutkimuksen kohde, populaatio, alue ja tarkoitus
käytetyt tutkimusmenetelmät (mikäli tutkimus on luonteeltaan teoreettinen
ja tiettyyn kirjalliseen materiaaliin, on mainittava tärkeimmät lähdeteokset;
mikäli on luonteeltaan empiirinen, on mainittava käytetyt metodit)
keskeiset tutkimustulokset tulosten perusteella tehdyt päätelmät ja
toimenpidesuositukset asiasanat
\end{abstract}

% Second abstract in english
% (in case the document is written in non-english)
% comment if not needed

\begin{abstracten}
If your thesis is in english this might also be required???
\end{abstracten}


% empty pagestyle for table of contents etc.
% otherwise you'll get simple page style with roman page numbers
\pagestyle{empty}

% mandatory
\tableofcontents

% if you want a list of figures
\listoffigures

% if you want a list of tables
\listoftables

% if you want a list of acronyms
\listofacronyms

% change the name if the default doesn't sound right
\renewcommand{\algorithmname}{Ohjelmalistaus}% 'list of X'
%   - it is also possible to define custom lists
%     \chapter{List Of Xs} 
%   - the secnumdepth trickery is needed because acronyms are as a
%     standard chapter and we are faking '\listofacronyms'
%
%\setcounter{secnumdepth}{-1}
%\input{your acronym chapter's file name}
%\setcounter{secnumdepth}{2}% setup page numbering, page counter, etc.%
\begin{comment}
The thesis starts here.

To better organize things, create a new tex file for each chapter
and input it below.

Avoid using the å, ä, ö or <space> characters in referred names and
underscores \_ in file names (may break hyperref).

Good luck!
\end{comment}

\input{johdanto.tex}\input{toinenluku.tex}

%\input{file_name_of_chapter_x}
%\input{file_name_of_chapter_y}

\begin{comment}
The thesis main content ends here.
\end{comment}
\printbibliography

\begin{comment}
Important! Create the appendix chapters with command \textbackslash appchapter\{some
name\} instead of \textbackslash chapter\{some name\} for the automagic
page counting to work!
\end{comment}


\appchapter{Liitedokumentti}

Liitteen ohjelmakoodi \ref{alg:Tyyppiluokka-Monad} kuvaa matemaattisen
monadirakenteen pohjalta rakentuvan Haskellin tyyppiluokan. Tyyppiluokan
voi nähdä eräänlaisena abstraktina ohjelmointirajapintana (API\nomenclature[API]{API}{Application Programming Interface}),
joka muodostaa ohjelmoijalle abstraktin ohjelmointikielen käyttöliittymän
(UI\nomenclature[UI]{UI}{User Interface}).

\begin{algorithm}[tbh]
\begin{minted}{haskell}
class Monad m where
    ( >>= )         :: m a -> (a -> m b) -> m b
    return          :: a                 -> m a

    fail            :: String            -> m a
    (>>)            :: m a -> m b        -> m b
    m >> k          =  m >>= \_ -> k       -- default

instance Monad IO where  ...               -- omitted
\end{minted}

\caption{Tyyppiluokka 'Monad'.\label{alg:Tyyppiluokka-Monad}}
\end{algorithm}

\newpage{}

Ensimmäisen liitteen toinen sivu. Ohjelmakoodi \ref{alg:Monadin-k=0000E4ytt=0000F6=0000E4.}
demonstroi vielä monadin käyttöä.

\begin{algorithm}[tbh]
\begin{minted}{haskell}
main =
return "Your name:" >>=
putStr >>=
\_ -> getLine >>=
\n -> putStrLn ("Hey " ++ n)
\end{minted}

\caption{Monadin käyttöä.\label{alg:Monadin-k=0000E4ytt=0000F6=0000E4.}}
\end{algorithm}


\appchapter{Liitedokumentti 2}

Tässä esimerkki\pagebreak{}

toisesta kaksisivuisesta liitteestä.
\end{document}
